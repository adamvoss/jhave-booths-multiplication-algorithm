\documentclass{acm_proc_article-sp}

\usepackage[utf8]{inputenc}

\begin{document}

\title{Booth's Multiplication}

\numberofauthors{2}
\author{
    \alignauthor
    Christopeher W. Jenkins\\
       \affaddr{Trinity College}\\
       \affaddr{One Trinity Place}\\
       \affaddr{San Antonio, TX 78212-7200}\\
       \email{cjenkin1@trinity.edu}
    \alignauthor
    Adam D. Voss\\
       \affaddr{Luther College}\\
       \affaddr{700 College Drive}\\
       \affaddr{Decorah, IA 52101}\\
       \email{vossad01@luther.edu}
}

\maketitle

\begin{abstract}
//TODO abstract
\end{abstract}

\section{Introduction}
Computer design has been described as the art by which a use or need for computation is mapped onto economic, temporal and technological constraints \cite{ibm370}.  It is therefore appropriate that any education of computer design and computer organization keep this paradigm in focus.  At the same time, due to a limit on the time in undergraduate courses, it is general beneficial to teach students at a more abstract and higher level view of computer architecture.  A good example of theres consideration are the process by which two binary values are multiplied.  Issues of efficiency, special hardware considerations, and the problem of handling signed binary values must be taken into account from a design standpoint.  However, the algorithm for binary number multiplication is of sufficient complexity to allow the student to view it in terms of a hardware's instruction set architecture, instead of at the logic-circuit level.

An effective method for the multiplication of signed binary values was introduced by Andrew Booth in 1951. Unlike other methods, it involved the comparison of pairs of bits as the algorithm is executed.  At its heart lies the observation that a series of powers of two $2^{n+1} - 2^{n-k}$. The most important result of this method was that little extra hardware or processing needed to be used to derive the correct result for all cases\cite{needsCitation}. Another result of this method is that in many cases significant speedup can be achieved due to a reduced number of arithmetic operations. For its clever use of mathematical insight into a problem to meet design requirements, we deemed Booth's method an appropriate subject for out research\cite{booth}. %That last sentence should not have a citation

Many textbooks include figures depicting the execution of binary multiplication algorithms.  We have chosen to include with our hypertextbook an algorithm visualization, developed for the Java Hosted Algorithm Visualization Environment (JHAVÉ). The motivation behind this desire was three-fold: first we felt an explicit step-by-step demonstration of the running algorithm would preffred by the student over a state diagram; second, the JHAVÉ framework provides tools for automated assessment in the form of questions generated during the algorithm visualization; finally, the algorithm framework enabled the creation and assessment of additional exercises associated with the textbook, through its "input generator" feature.  This is in keeping with the current research in the field of algorithm visualization\cite{needsCitation}.  The Innovation and Technology in Computer Science Education Working group (ITiCSE) has shown the effectiveness of visualization is largely dependant on the level of engagement of the learner\cite{needsCitation}.  They have accordingly developed an engagement taxonomy, the five levels of which are directly correlated with our three listed motivations.  The levels of engagement our visualization supports are the Following: Viewing, Responding, and Changing, as defined by ITiSCE.

\balancecolumns
\end{document}
