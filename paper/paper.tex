\documentclass{acm_proc_article-sp}

\usepackage[utf8]{inputenc}
\usepackage{biblatex}

\bibstyle{abbrv}
\bibliography{pbib}

\begin{document}

\title{An Effective Pedagogical Module for Booth's Multiplication Algorithm}

\numberofauthors{2}
\author{
    \alignauthor
    Christopher W.
Jenkins\\
       \affaddr{Trinity University}\\
       \affaddr{One Trinity Place}\\
       \affaddr{San Antonio, TX 78212-7200}\\
       \email{cjenkin1@trinity.edu}
    \alignauthor
    Adam D.
Voss\\
       \affaddr{Luther College}\\
       \affaddr{700 College Drive}\\
       \affaddr{Decorah, IA 52101}\\
       \email{vossad01@luther.edu}
}

\maketitle

\begin{abstract}
//TODO abstract
\end{abstract}

\section{Introduction}
In teaching computer design, it is important to discuss the process by which binary numbers are multiplied at the machine level.
However, there is variation in how this subject is covered, for example over which method(s) of multiplication to discuss, how negative numbers are dealt with, how efficiency is achieved, and how much hardware-level detail should be included in the explanation of the process.
Additionally, there is always a trade-off between the time spent in class and the amount of material covered.
Because of this, the professor may choose to spend less class time discussing methods for binary multiplication in order to cover other topics.

%constructivism?
We propose that an effective, self-contained learning module available online could resolve these issues.
The module we designed for this purpose is composed of an online text resource for the subject and an algorithm visualization.
As current research in the field of algorithm visualizations suggests that the most important aspect of an effective algorithm visualization is the level of user engagement\cite{tnaps:visengage}, we developed questions and exercises to go along with the module.
We use the engagement taxonomy developed by the Innovation and Technology in Computer Science Education Working Group in 2002 (ITiCSE02) to describe the different components of our visualization.


An effective method for the multiplication of signed binary values was introduced by Andrew Booth in 1951.
Unlike other methods, it involves the comparison of pairs of bits as the algorithm is executed.
At its heart lies the observation that a series of powers of two $2^n + 2^{n-1} + ...
+ 2^{n-k}$ is equal to $2^{n+1} - 2^{n-k}$.
The most important contribution of this method is that little hardware or processing is needed to derive the correct result for all cases\cite{booth}.
Another result of this method is that in many cases significant speedup over other methods of binary multiplication can be achieved due to a reduced number of arithmetic operations.\cite{needsCitation}
For its use of mathematical insight into a problem to meet design, requirements, we believe that Booth's algorithm is of educational value and chose to use it in our hypertext and visualization.
However, while some textbooks include a discussion of Booth's algorithm in their chapter(s) regarding binary multiplication, it is not universally discussed in textbooks, and furthermore because of the time constraints mentioned above a professor may choose to gloss over or elide the topic completely.

Many textbooks include figures depicting the execution of the binary multiplication algorithm(s) along with their description.
Such a figure can be considered a kind of algorithm visualization, and as such lack an effective form of engagement other than primitive "Viewing".
It would be beneficial if, for example, the student were able to view multiple visualizations with hand-picked values.
For this reason we have chosen to include with our hyper-textbook an algorithm visualization, developed for the Java Hosted Algorithm Visualization Environment (JHAVÉ)\cite{JHAVE}.
The motivation behind this decision was three-fold: first, we felt that an interactive and step-by-step demonstration of the running algorithm would be preferred by the student over a static diagram; second, the JHAVÉ framework provides tools for automated assessment in the form of questions generated during the visualization; finally, through the use of the framework's "input generator" feature, we were both able to allow students to create their own test data for the visualization and to create additional exercises which require the student to input data which will cause specific results during or on the completion of the algorithm.
These three motivations are directly correlated to three levels of the engagement taxonomy defined by ITiCSE02: "Viewing", "Responding", and "Changing".

\section{Implementation of the Visualization}
In our design and implementation of the visualization for Booth's algorithm, our primary consideration was for active user engagement.
Prior research by ITiSCE suggests that this is the most important aspect for an effective visualization.\cite{tnaps:visengage}
Organized by level of engagement, our visualization supports the following: at the "Viewing" level, it includes graphical representations of registers and binary arithmetic, text annotations and pseudo code
 display; at the level of "Responding", we include pop-up questions designed to check the user's basic understanding of the algorithm; and at the "Changing" level, we designed a set of exercises in which the user is given a constraint or set of constraints on the behavior of the executing algorithm and must provide data to meet that constraint.
In the standard use of the visualization, we also allow the user to choose their own input data or use the provided default values.

JHAVÉ was chosen as the system to develop a visualization for Booth's algorithm, as it supported these features and was in fact designed with these three levels of engagement in mind.\cite{JHAVE}

\subsection{Standard Mode}
Our visualization was designed so that the user would step through each of the major operations of Booth's multiplication algorithm.
For each of these operations, there is a corresponding text annotation at the top of the screen and highlighted line of pseudo code in the right pane.
For addition and subtraction operations, we provide space on the right side of the screen, designated in the visualization as "Math/ALU", to reinforce binary arithmetic.
We also chose to keep a running history of the execution of the algorithm on screen for the user's reference.
Registers in history are grayed out so as to distinguish them from the registers which are active in the algorithm.
\subsection{Questions}
JHAVÉ also supports the inclusion of questions which appear during the visualization of the algorithm.
These were created to test the user on multiple levels of understanding, as defined by Bloom's taxonomy(NEEDS JUSTIFICATION).
These questions are strategically placed and appear approximately once every -- snapshots to ensure the user in engaged at important moments of the algorithm's execution.
To prevent boredom in the user and predictability in the questions, multiple presentations of the same "type" of question are used.
Questions are in the following formats: true or false, fill in the blank, multiple choice and multiple selection.
\subsection{Exercises}%potentially something subject to change.
Finally, through the use of input generators to JHAVÉ, we included several exercises to go along with the visualization.
The user is given certain constraints, in the form of an expected behavior in or result of the algorithm and is asked to provide data which will meet those constraints.
There are -- exercises available to the student.
Exercise modes will not include questions, but instead a status message accepting or rejecting the user's input at the end of the visualization.

\printbibliography

\balancecolumns
\end{document}
