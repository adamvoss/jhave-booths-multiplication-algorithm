\documentclass{acm_proc_article-sp}

\usepackage[utf8]{inputenc}

\begin{document}

\title{Booth's Multiplication}

\numberofauthors{2}
\author{
    \alignauthor
    Christopher W.
Jenkins\\
       \affaddr{Trinity University}\\
       \affaddr{One Trinity Place}\\
       \affaddr{San Antonio, TX 78212-7200}\\
       \email{cjenkin1@trinity.edu}
    \alignauthor
    Adam D.
Voss\\
       \affaddr{Luther College}\\
       \affaddr{700 College Drive}\\
       \affaddr{Decorah, IA 52101}\\
       \email{vossad01@luther.edu}
}

\maketitle

\begin{abstract}
//TODO abstract
\end{abstract}

\section{Introduction}
Computer design has been described as the art by which a use or need for computation is mapped onto economic, temporal and technological constraints \cite{ibm370}.
It is therefore appropriate that any education of computer design and computer organization keep this paradigm in focus.
At the same time, due to a limit on the time in undergraduate courses, it is general beneficial to teach students at a more abstract and higher level view of computer architecture.
A good example of theres consideration are the process by which two binary values are multiplied.
Issues of efficiency, special hardware considerations, and the problem of handling signed binary values must be taken into account from a design standpoint.
However, the algorithm for binary number multiplication is of sufficient complexity to allow the student to view it in terms of a hardware's instruction set architecture, instead of at the logic-circuit level.

An effective method for the multiplication of signed binary values was introduced by Andrew Booth in 1951.
Unlike other methods, it involved the comparison of pairs of bits as the algorithm is executed.
At its heart lies the observation that a series of powers of two $2^{n+1} - 2^{n-k}$.
The most important result of this method was that little extra hardware or processing needed to be used to derive the correct result for all cases\cite{needsCitation}.
Another result of this method is that in many cases significant speedup can be achieved due to a reduced number of arithmetic operations.
For its clever use of mathematical insight into a problem to meet design requirements, we deemed Booth's method an appropriate subject for out research\cite{booth}.
%That last sentence should not have a citation

Many textbooks include figures depicting the execution of binary multiplication algorithms.
We have chosen to include with our hypertextbook an algorithm visualization, developed for the Java Hosted Algorithm Visualization Environment (JHAVÉ).
The motivation behind this desire was three-fold: first we felt an explicit step-by-step demonstration of the running algorithm would preferred by the student over a static diagram; second, the JHAVÉ framework provides tools for automated assessment in the form of questions generated during the algorithm visualization; finally, the algorithm framework enabled the creation and assessment of additional exercises associated with the textbook, through its "input generator" feature.
This is in keeping with the current research in the field of algorithm visualization\cite{needsCitation}.
The Innovation and Technology in Computer Science Education Working group (ITiCSE) has shown the effectiveness of visualization is largely dependant on the level of engagement of the learner\cite{needsCitation}.
They have accordingly developed an engagement taxonomy, the five levels of which are directly correlated with our three listed motivations.
The levels of engagement our visualization supports are the following: Viewing, Responding, and Changing, as defined by ITiSCE.

\section{Implementation of the Visualization}
In our design and implementation of the visualization for Booth's algorithm, our primary consideration was for active user engagement.
Prior research by ITiSCE suggests that this is the most important aspect for an effective visualization.\cite{ITiSCE}
Organized by level of engagement, our visualization supports the following: at the "Viewing" level, it includes graphical representations of registers and binary arithmetic, text annotations and pseudo code
 display; at the level of "Responding", we include pop-up questions designed to check the user's basic understanding of the algorithm; and at the "Changing" level, we designed a set of exercises in which the user is given a constraint or set of constraints on the behavior of the executing algorithm and must provide data to meet that constraint.
In the standard use of the visualization, we also allow the user to choose their own input data or use the provided default values.

JHAVÉ was chosen as the system to develop a visualization for Booth's algorithm, as it supported these features and was in fact designed with these three levels of engagement in mind.\cite{JHAVE}

\subsection{Standard Mode}
Our visualization was designed so that the user would step through each of the major operations of Booth's multiplication algorithm.
For each of these operations, there is a corresponding text annotation at the top of the screen and highlighted line of pseudo code in the right pane.
For addition and subtraction operations, we provide space on the right side of the screen, designated in the visualization as "Math/ALU", to reinforce binary arithmetic.
We also chose to keep a running history of the execution of the algorithm on screen for the user's reference.
Registers in history are grayed out so as to distinguish them from the registers which are active in the algorithm.
\subsection{Questions}
JHAVÉ also supports the inclusion of questions which appear during the visualization of the algorithm.
These were created to test the user on multiple levels of understanding, as defined by Bloom's taxonomy(NEEDS JUSTIFICATION).
These questions are strategically placed and appear approximately once every -- snapshots to ensure the user in engaged at important moments of the algorithm's execution.
To prevent boredom in the user and predictability in the questions, multiple presentations of the same "type" of question are used.
Questions are in the following formats: true or false, fill in the blank, multiple choice and multiple selection.
\subsection{Exercises}%potentially something subject to change.
Finally, through the use of input generators to JHAVÉ, we included several exercises to go along with the visualization.
The user is given certain constraints, in the form of an expected behavior in or result of the algorithm and is asked to provide data which will meet those constraints.
There are -- exercises available to the student.
Exercise modes will not include questions, but instead a status message accepting or rejecting the user's input at the end of the visualization.
\balancecolumns
\end{document}
